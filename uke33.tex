\documentclass[10pt,english]{article}
\usepackage[T1]{fontenc}
\usepackage[utf8]{inputenc}
\usepackage{amsmath}
\usepackage{mathtools}
\usepackage[margin=1in]{geometry}
\usepackage{listings}
\usepackage{graphicx}
\usepackage{float}
\renewcommand{\vec}[1]{\mathbf{#1}}
\newcommand{\problem}[1]{\subsection*{#1}}
\newcommand{\subproblem}[1]{\subsubsection*{#1}}
\newcommand{\rscript}[1]{\lstinputlisting[language=R]{#1.r}}
\newcommand{\xni}[1]{\bar{x}_{#1\cdot}}
\newcommand{\Xni}[1]{\bar{X}_{#1\cdot}}
\newcommand{\exfig}[1]{
\begin{figure}[H]
\includegraphics{#1}
\end{figure}
}
\newcommand{\avg}[1]{\bar{#1}}
\newcommand{\med}[1]{med{x}}
\newcommand{\wave}[1]{\tilde{#1}}
\newcommand{\dx}[0]{dx}
\newcommand{\set}[1]{\left\{ #1 \right \}}
\newcommand{\pardiff}[2]{\frac{\partial #1}{\partial #2}}
%%% Local Variables:
%%% mode: latex
%%% TeX-master: t
%%% End:

\begin{document}
\problem{A.1.1}
\subproblem{a}
Ja
\subproblem{b}
Ja (må omskrives)
\subproblem{c}
Nei (kvadratledd)
\subproblem{d}
Ja
\subproblem{e}
Nei, produktledd
\problem{A.1.2}
\subproblem{a}
\begin{align*}
\vec{x} -5\vec{y}&=(1,2,3,4)-5(10,1,0,4)\\
                 &=(1-50,2-5, 3, 4-20)\\
                 &=(49, -3, 3, -16)\\
\end{align*}
\subproblem{b}
\begin{align*}
  \frac{3}{2}\vec{z} &= \frac{3}{2}(12,7,3)\\
                     &= (18, \frac{21}{2}, \frac{9}{2})
\end{align*}
\subproblem{c}
$\vec{x}-5\vec{y}+\frac{3}{2}\vec{z}$, ikke mulig grunnet forskjellig antall dimensjoner
\subproblem{d}
\begin{align*}
  \vec{x}\cdot 3 \vec{y}&=(1,2,3,4)\cdot (30,3,0,12)\\
                        &=(1,6,0,48)\\
\end{align*}
\problem{A.1.3}
\begin{align*}
  (x_1, x_2, x_3) &=  (4,3,7)\\
  \\
  (x_1, x_2, x_3, x_4, x_5) &= (1,18,2,9,3)\\
  \\
  \vec{x} &= (2,0,3)\\
  \vec{y} &= (-3, 0, 2)\\
  \\
  \vec{x} &= (2,1,3)\\
  \vec{y} &= (2,-5,1)\\
\end{align*}
\problem{A.1.4}
\subproblem{a}
\begin{align*}
  7x-5x &= 3\\
  2x &= 3\\
  x &= \frac{3}{2}
\end{align*}
\subproblem{b}
\begin{align*}
  7x-5y &= 3\\
  7x-5s_1&= 3\\
  7x &= 3+5s_1\\
  x &= \frac{3}{7} + \frac{5}{7}s_1\\
  \\
  &\set{(\frac{3}{7} + \frac{5}{7}s_1, s_1) : s_1 \in R}\\
\end{align*}
\subproblem{c}
\begin{align*}
  7x-5y &= 3\\
  x+5y &= 1\\
  8x &= 4\\
  x &= \frac{1}{2}\\
\\
  \frac{1}{2} + 5y &= 1\\
  5y&= \frac{1}{2}\\
  y&= \frac{1}{10}\\
\end{align*}
\subproblem{c}
\begin{align*}
  y &= 2-2x=2(1-x)\\
\\
  2x + 3y &= 8\\
  2x + 3\cdot 2(1-x) &= 8\\
  x + 3-3x &= 4\\
  -2x &= 1\\
  x &= -\frac{1}{2}\\
  y &= 2-2(-\frac{1}{2} = 3\\
  \\
  x+y+z &= 3\\
  z &= 3-x-y = 3+\frac{1}{2}-3=\frac{1}{2}\\
\end{align*}
\problem{A.1.5}
\subproblem{a}
\begin{align*}
  x+y+x &= 0\\
  x+y-x &= 0\\
  z &= 0\\
  x+y&=0\\
\\
  (x,y,z) &\in \set{(s_1, -s_1, 0) : s_1 \in R}\\
\end{align*}
\subproblem{b}
\begin{align*}
  x+y-z &= 3\\
  2x+2y-2z &= 6\\
  \\
  (x,y,z) &\in \set{(3+s_1-s_2, s_2, s_1) : s_1, s_2 \in R}\\ 
\end{align*}
\subproblem{c}
\begin{align*}
  x+y+z+w &= 15\\
  x-2y+4z-w &= 12\\
  2x-y+5z &= 3\\
  y&= -3+5s_3+2s_1\\
  \\
  (x,y,z,w) &\in \set{(s_1, -3+5s_3+2s_1, s_3, 15-s_1-s_3+3-5s_3-2s_1): s_1,s_3 \in R}\\
  (x,y,z,w) &\in \set{(s_1, -3+5s_3+2s_1, s_3, 18-6s_3-3s_1): s_1,s_3 \in R}              
\end{align*}
\subproblem{d}
\begin{align*}
  x-2y+2z &= 4\\
  2x-4y+4z &= 2\\
  \\
  2x-4y+4z &= 8\\
\end{align*}
Ukompatibelt
\problem{A.1.6}
\subproblem{a}
\begin{align*}
  3x_1 - 5x_2 +4x_3 &= 7\\
  x_1 &= \frac{7}{3}+\frac{5}{3}s_1- \frac{4}{3}s_2\\
  (x_1, x_2, x_3) &\in \set{(\frac{7}{3}+\frac{5}{3}s_1- \frac{4}{3}s_2, s_1,s_2) : s_1, s_2 \in R}
\end{align*}
\subproblem{b}
\begin{align*}
  L_1&:&x_1 + x_2 +2x_3 &= 8\\
  L_2&:&-x_1 -2x_2 +3x_3 &= 1\\
  L_3&:&3x_1-7x_2 +4x_3 &= 10\\
  L_1+L_2&:& -x_2+5x_3 &= 9\\
     && x_2       &=5x_3- 9\\
  2L_1+L_2&:&x_1+7x_3 &=17\\
     &&x_1       &=17-7x_3\\
  L_3 &:& 51-21x_3-35x_3+63+4x_3 &= 10\\
  && 104-52x_3&=0\\
  && x_3 &= 2\\
  L_1+L_2 &:&  x_2&=5x_3-9=1\\
  2L_1-L_2 &:& x_1&=17-7x_3=3\\
  \\
\end{align*}
\subproblem{c}
\begin{align*}
  L_1 &:& 2x-3y&=-2\\
  L_2 &:& 2x + y &=1 \\
  L_3 &:& 3x+2y &= 1\\
  L_2-L_1 &:& 4y &=3\\
  && y = \frac{3}{4}\\
  L_3-\frac{3}{2}L_2 &:& \frac{1}{2}y &= -\frac{1}{2}\\
  &&y &=-1\\
\end{align*}
Ligningsettet er inkompatibelt
\subproblem{d}
\begin{align*}
  L_1 &:& 2x_1+2x_2+2x_3 &= 0 \\
  L_2 &:& -2x_1 +5x_2+2x_3 &= 1\\
  L_3 &:& 8x_1 + x_2 + 4x_3 &= -1\\
  3 L_1 -L_2 &:& 8x_1 + x_2 + 5x_3 &=1\\
  \\
  L_1 + L_2 &:& 7x_2 + 4x_3 = 1\\
      && x_3 = \frac{1}{4}(1-7x_2)\\
  L_1 &:& x_1 + x_2 + x_3 &= 0\\
  L_1 &:& x_1 &= -x_2-x_3\\
      && x_1 &= -x_2-\frac{1}{4}+\frac{7}{4}x_2 = \frac{3}{4}x_2-\frac{1}{4}\\
      && x_1 &= 3s-\frac{1}{4}\\
      && x_2 &= 4s\\
      && x_3 &= \frac{1}{4}-7s\\
  &&(x_1,x_2,x_3) &\in \set{(3s-\frac{1}{4}, 4s, \frac{1}{4}-7s) : s \in R}
\end{align*}
\subproblem{e}
\begin{align*}
  L_4&:& 3x-3w &= -3\\
  x-w &= -1\\
  L_3-2L_2 &:& -5x+5w &= 1\\
  x-w &= -\frac{1}{5}\\
\end{align*}
Ligningsettet er inkompatibelt
\problem{A.1.7}
\problem{A.1.8}
\begin{itemize}
\item To av linjene er parallelle: Ingen løsninger
\item Linjene sljærer hverandre i samme punkt: En løsning
\item Linjene skjærer hverandre i forskjellig punkt: Ingen løsning
\item Alle tre linjene sammenfaller: Uendelig mange løsninger
\end{document}
\problem{A.1.10}


%%% Local Variables:
%%% mode: latex
%%% TeX-master: t
%%% End:
